\chapter{Introduction}
\pagenumbering{arabic}


\section{Motivation}
Among the OECD member countries, Norway has the highest rate of employment for young people with higher education (25-34 years). The employment rate for this population is of 91\% for people with a bachelor's degree and 95\% for those with a master's degree \parencite{fazli_hoy_2022}. From this age, and until retirement this population will be working a considerable amount of years. Additionally, due to the rise of pension costs and living expectancy, the retirement age of about 67 years presently \parencite{noauthor_pensjonsalder_nodate} will likely increase. A retirement committee recently propositioned that the retirement age be increased regularly in the future \parencite{haugan_pensjonsutvalget_2022}.

With an expected employment duration of between 40 and 50 years for the present generation. It is a given that workplace well-being is primordial in living a good, healthy and happy life. Additionally, due to the recent pandemic and the stress it put on workers. An increase of the burnout rates has taken place around the globe. 

According to the Work and Well-being Survey of US adult workers by the \textit{American Psychological Association}, '3 in 5 employees reported negative impacts of work-related stress' in 2021 \parencite{abramson_burnout_2022}. In a survey done in Norway by \textit{Forskerforbundet} in 2021, 56\% of the respondents were not happy with the new working situation. The shift to teaching digitally, mixed with working from home added an increased workload. This added workload was reported by 40\% of the researchers, where only 16\% were compensated for it \parencite{svarstad_tarene_2021}. Similarly, in a longitudinal study carried out by \textit{Nasjonalt Kunnskapssenter om vold og traumatisk stress}, on hospital workers during the pandemic. They found that 20\% of the respondents reported significant anxiety and depression symptoms, and that one in six workers, about 16\%, reported burnout symptoms \parencite{noauthor_studie_2022}.

Not surprisingly, work movements like The Great Resignation and "quiet quitting" have arisen post-pandemic. A global increase in resignations is now known as The Great Resignation. The US reached a 20-year high "quit rate" in November 2021, where research by the \textit{Pew Reasearch Center} found that the main reasons for quitting was "low pay, a lack of opportunities for advancement and feeling disrespected at work" \parencite{parker_majority_2022}. Italy has seen an increase of 10\% in resignations in the middle of 2021, compared to 2019. Even quit rates in the tech-industry in India are at around 22\%, corresponding to a million of workers quitting at the end of 2021 \parencite{armillei_si_2021}. Another trend called "quiet quitting", started with TikTok. With "quiet quitting", workers are encouraged to prevent being overworked, instead achieving a greater work-life balance by only doing the job requirements, not going above and beyond \parencite{rosalsky_economics_2022}.

For all of these reasons, putting a spotlight on burnout; how it arises, how it can be prevented and how to get back from it, is important. Furthermore, enlightening the topic with the help of a serious game, will hopefully aid in preventing burnout, in an engaging and fun way.

\section{Context}
This project has been developed in collaboration with the Department of Computer Science at the Norwegian University of Science and Technology. The work is in foundation of an upcoming master thesis developing a serious game aiming to enlighten the workforce on burnout. 

\section{Research Questions}

As burnout can arise in a multitude of professions, to focus the research, this project will 

The research questions that this project will look into are the following:

\begin{itemize}
    \item[] \textbf{RQ1}: How can informative serious games be used to raise awareness around burnout in the IT profession?

To answer the research question, 
    \item[] \textbf{RQ1.1}: What kind of awareness about burnout do IT professionals have?
    \item[] \textbf{RQ1.2}: What type of burnout scenarios are relevant  for serious games?
\end{itemize}

\section{Results}

\section{Report Outline}
I think it will end up looking something like this
% TODO change at the end
\begin{itemize}
    \item introduction
    \item problem definition
    \item literature review (?)
    \item interviews
    \item discussion
    \item conclusion
\end{itemize}