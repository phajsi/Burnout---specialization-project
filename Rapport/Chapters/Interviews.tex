\chapter{Interviews with professionals in the IT industry}

Semi-structured interviews with X participants were conducted to gain insight into the burnout awareness of IT professionals.

\section{Purpose}
To answer RQs \textbf{1.1}, \textbf{1.2} and \textbf{1.3}: "What is burnout in the IT industry?", "What burnout awareness do IT professionals have?" and "What does the IT industry do to prevent burnout?" 6(?) semi-structured interviews were conducted. 

Interviewing professionals with years of experience in the IT-industry, allows for examining the experiences that the interviewees have on burnout prevention activities in the workplace. It also explores the views they have on what kind of information they would like to know about burnout, and how a future serious game can be helpful.

\section{Participants}
%I suggest you start writing the method section in your report. One of the sub-section will be about the selection of participants containing an 

The participants consisted on (...). 

\section{Method}
This section describes the preparations made before the interviews, the tools used, and how the interviewees were selected and conducted.

\subsection{Interview method}
The semi-structured interview method was chosen before conducting the interviews. 

In semi-structured interviews the interviewer prepares a list of themes and questions to be covered, but they are also able to change the order of questions, adapting to the flow of the conversation with the possibility of asking additional questions. Semi-structured interviews  allow interviewees to "speak their mind", they are used if the primary purpose is "discovery" and in-depth investigations \parencite[188]{oates_researching_2006}. 

This method is widely used in qualitative research, it allows for having a template (Appendix \ref{Intervjumal}) as a start-point for conducting the interview, but also allows for follow-up questions. This gives both interview and interviewee the possibility of discussing topics in-depth, without overlooking other relevant questions. 

\subsection{Participant selection}
In order to get insight into the awareness of burnout that professionals in IT have, and to understand what the industry does to prevent burnout, participants in specific roles were selected to be interviewed. The professionals selected were either people with experience in being in charge of a team, like team-leaders, managers and tech-leads, and people working in Human Resources (HR). 

The team-leader role was selected because studies have shown that supervisor support influenced the three facets of burnout: job efficacy, work exhaustion and professional efficacy for the better \parencite[9]{zaza_drivers_2021}. Additionally, in being in charge of a team and it's success, a part of the successful management of a team encompasses the well-being of it's individuals and therefore, insight into burnout awareness from the point of view of team-leaders was very important. 

Insight from people working in HR is equally important, since HR is the part of companies which main focus are it's employees. HR can train managers on mentoring, team building and employee recognition among other things. HR management can include initiatives that prevent burnout across the company. 

%explanation of the rationale for your selection (why exactly these roles and these type of companies). Is there something in what you have read that supports your decision? What makes you think that it is worth to go and check if things are similar/different for these two types of companies?  I am not challenging your decision. I think it makes sense, but you need to explain and justify it. Also remember that, based on your rationale, there might be specific questions to ask. Especially when things turn out to be different, it might be useful to have questions that help you to understand WHY. 

%You say you want to interview men and women.  This is OK. Do you see gender issues as an important element that is worth to investigate? Considering the gender gap in IT industry, there might be biases and stereotypes that make women struggle more.

%When it comes to the type of companies, you might also consider the public sector (you do not necessarily have to do it, but just keep it in mind).

\subsection{Recording and transcription}

\section{Interview pilot}
In order to test the comprehensibility of the questions planed (Appendix \ref{Intervjumal}) prior to starting with the official interviews, a pilot test was conducted. A fellow student in Computer Science who had ample experience in working as an IT-professional, in both internships and part-time jobs, was asked to participate. 

\subsection{Results}
No recordings were taken, only notes were taken during the interview and after.

From the pilot test ...

\section{Interview analysis}

\section{Results}

\section{Conclusion}
\section{Summary}