\chapter{Problem Definition}  

% 1 - overview of the issue discussed in the paper, as well as its background and who it affects. It may also describe your objectives in performing your research. %Who acknowledges burnout officially - START
The \textit{World Health Organization}, WHO, added burnout to the \textit{International Classification of Diseases}, the ICD-11, in 2019. In it, burnout was defined as a syndrome resulting from "chronic workplace stress that has not been successfully managed" \parencite{who_notitle_2019}. 

% 2 - When research on burnout started - BACKGROUND
This recent addition was possible because of the research on burnout by Dr. Christina Maslach, the pioneer researcher on burnout. Burnout as a research field began in the 1970s, trying to make sense of the experience some workers got when they started a new job with enthusiasm and energy, only to over time end up disillusioned; and with feelings of exhaustion, frustration, anger and cynicism \parencite[38]{maslach_understanding_2017}.

\section{Three burnout dimensions}

From the initial research on the topic, common themes arrived that concluded with the three basic dimensions of the burnout experience: "an overwhelming exhaustion, feelings of cynicism and detachment from the job, and a sense of ineffectiveness and lack of accomplishment" %from (see Maslach, 1993). 
Maslach then developed the Maslach Burnout Inventory, together with a General Survey, to assess these three burnout dimensions in her research \parencite[38]{maslach_understanding_2017}.

Later, at the beginning og the twenty-first century, researchers have also looked into the opposite of burnout, identified as "engagement", but how exactly it relates to burnout is still debated \parencite[40]{maslach_understanding_2017}.

\textbf{Exhaustion}

Exhaustion reflects the strain dimension of burnout, the basic stress experienced by a person. It refers to the "feelings of being overextended and depleted of one’s emotional and physical resources" \parencite[41]{maslach_understanding_2017}. 

\textbf{Cynicism}

The cynicism dimension captures the individual's relationship to work, by assessing the negative, distancing response to their job and the people in it.

\textbf{Professional Inefficacy}

Professional inefficacy is about the sense of ineffectiveness and of being unsuccessful at work. It can lead to low morale, to becoming unproductive, and it to aggravate the feeling of being undeserving of their skills and accomplishments, also known as \textit{Imposter Syndrome} \parencite{benisek_what_2022} \parencite{yerbo_yerbo_nodate}.

Exhaustion is not viewed as unprofessional, there can be pride associated to exhaustion. But exhaustion can prompt actions to distance oneself from work, most likely as a coping mechanism to work overload, leading to cynicism. Cynicism in contrast is viewed as unprofessional, and therefore these feelings are most likely not shared with others and they contribute to a diminished professional efficacy. 

When an individual experiences burnout with all its facets, they loose a "a psychological connection with their work that has implications for their motivation and their identity" \parencite[41]{maslach_understanding_2017}.

\section{Areas of Worklife Scale model}
The research on burnout has identified organizational risk factors across many occupations in various countries. Maslach and Leiter analyzed the research literature and identified six areas of worklife most relevant for the relationship that people develop with their work: workload, control, reward, community, fairness and values \parencite{brom_areas_2015}.

\textbf{Workload}\\
When job demands surpass human limits, the most likely consequence is emotional exhaustion. People doing too much in a short amount of time, with limited resources \parencites{brom_areas_2015}[44]{maslach_understanding_2017}.

\textbf{Control}\\
The control dimension includes the individual's perceived capacity to influence decisions related to their work, along with their ability to exercise personal autonomy and to access resources like social support, to complete their work \parencites{brom_areas_2015}.

\textbf{Rewards}\\
Rewards refers to the extent that workers perceive that rewards; both monetary, social and intrinsic rewards, are consistent with their personal expectations. If people feel they are neglected by the employer's material and social reward system, they will feel out of sync with its values \parencites{brom_areas_2015}.

\textbf{Community}\\
The community dimension assesses the overall quality of social interaction at work and the sense of community in an organization. It encompasses interpersonal conflicts, informal social support, closeness and the capacity to work as a team. Supervisor support is more consistently associated with exhaustion; reflecting the supervisors' impact on employees' workload, while coworker support is related to accomplishment or efficacy \parencites{brom_areas_2015}[46]{maslach_understanding_2017}.

\textbf{Fairness}\\
Fairness encompasses the extent to which decisions and resource allocation at work are perceived as fair \parencite{brom_areas_2015}

\textbf{Values}\\
Finally, the values dimension cover the ideals and motivation that appeal people to their jobs. They are therefore the motivating connection between the individual and the workplace. A value conflict in the job can be: that the worker makes a trade-off between work they want to do and work they do; that personal career aspirations or personal values are in conflict with the organizational values; or that the worker is caught between conflicting organizational values \parencites{brom_areas_2015}[47]{maslach_understanding_2017}.

% 3 - why burnout is important for the individual and workplaces - AFFECTS
\section{Outcomes of burnout}

Besides the negative burnout dimensions, the problem with burnout is it's role as a mediator of negative personal and professional outcomes.

\textbf{Work}

Burned out employees have greater job dissatisfaction, are less committed, more absent and are more intent to leave their jobs, which not surprisingly leads to poorer work performance. They have a negative effect on their colleagues, causing more personal conflicts and disrupting job tasks. These negative effects can "spill" over on workers' home life. Burnout workers were rated in more negative ways by their spouses. The workers themselves have reported that their work has impacted negatively their family and that their marriage was unsatisfactory \parencite[49]{maslach_understanding_2017}. 

\textbf{Health}

The exhaustion facet of burnout correlates to typical stress symptoms as headaches, chronic fatigue, gastrointestinal disorders, muscle tension, hypertension, cold and flu episodes, and sleep disturbances. A longitudinal study done by Maslach and Leiter, found that workload and exhaustion predicted the rate of workplace injuries during the following year. Other research has also found a link between burnout and lifestyle practices with health risks, like smoking, alcohol use and psychotropic drug use.  %(Leiter et al., 2013). 
In the mental health aspect, burnout is predictive of depression, anxiety and irritability \parencite[50]{maslach_understanding_2017}. %(Greenglass and Burke, 1990; Schonfeld, 1989) 
% Work outcomes Burnout has been frequently associated with various forms of negative
% responses to the job, including job dissatisfaction, low organizational commitment, absenteeism,
% intention to leave the job, and turnover (see Schaufeli and Enzmann, 1998, for a
% review). People who are experiencing burnout can have a negative impact on their colleagues,
% both by causing greater personal conflict and by disrupting job tasks. Thus, burnout can be
% “contagious” and perpetuate itself through social interactions on the job (Bakker, Le Blanc,
% and Schaufeli, 2005; Gonz´alez-Morales, Peir´o, Rodr´ıguez, and Bliese, 2012). Such findings
% suggest that burnout should be considered as a characteristic of workgroups rather than simply
% an individual syndrome.
% Burnout can also have a negative “spillover” effect on workers’ home life. Workers experiencing
% burnout were rated by their spouses in more negative ways (Jackson and Maslach,
% 1982; Zedeck, Maslach, Mosier, and Skitka, 1988), and they themselves reported that their
% work has a negative impact on their family and that their marriage is unsatisfactory (Burke
% and Greenglass, 1989; 2001).