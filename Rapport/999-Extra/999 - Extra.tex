
\section{TEORI}:
% ORKLA TASTY NATURLI':
% Vann, soyaprotein, vegetabilsk olje (kokos, raps), løk, aroma, krydder, tomat, surhetsregulerende middel (kaliumlaktat, kaliumacetat), stabilisator(metylcellulose), fargestoff (rødbete), sjampinjong, fargestoff(karamell), hvitløk.

% HOFF LIV LAGA PLANTEBASERT BURGER ORIGINAL:
%Raspet potet(49%), brune linser (24%), aspargesbønner (7%), mais (7%), maissemule/mel, solsikkeolje, kikertmel, krydderbl [salt, mod.stivelse (potet), chili, løk, paprika, gjærekstrakt, hvitløk, spi-sskummen, jalapeno, karve, oregano, paprikaekstrakt, aroma (sitron, tomat), dektrose, erteprotein, hevemiddel (e450, e500), for- tykningsmiddel (xantangummi, guargummi), maisstivelse, potet- stivelse.

% BEYOND BURGER:
%Vann, erteproteinisolat (16 %), rapsolje, kokosolje, risprotein, smakstilsetninger, stabilisatorer (metylcellulose), potetmel, epleekstrakt, fargestoffer (røbetekstrakt), maltodekstrin, granatepleekstrakt, salt, kaliumklorid, sitronjuicekonsentrat, maiseddik, gulrotpulver, emulgator (solsikkelecitin).

% HALSAN KØK PLANT-BASE BURGER:
%Rehydrerat soja-/soyaprotein (58,5%), vatten/vann, rehydrerat vete-/hvete-/hvedeprotein (13%), vegetabilisk olja/olje (solros/solsikke, raps i varierande proportion), lök/løk, stärkelse/stivelse, stabiliseringsmedel/stabilisator (metylcellulosa, karragenan), salt, maltodextrin, jästextrakt/gjær-/gærekstrakt, aromer/aroma, lök-/løkpulver, vitlöks-/hvidløgspulver, maltextrakt (korn/bygg/byg), druvsocker/glukose/dextrose, färgämne/fargestoff (karamelliserat socker), rökt/røkt/røget maltodextrin, kryddor/krydder/krydderi, surhetsreglerande medel (citronsyra).

%%%%%%%%%%%%%%%%%%%%%%%%%%%%%%%%%%%%%

\begin{figure}[H]
    \centering
    \includegraphics[scale=1.3]{Figurer/Teori/Texture/ShearForceSetup.png}
    \caption{Setup for shearing test. Retrieved from \cite{garcia-segovia_texture_2011} with some modifications.}
    \label{fig:Teori:ShearingTestSetup}
\end{figure}
\begin{figure}[H]
    \centering
    \includegraphics[scale=0.9]{Figurer/Teori/Texture/ShearForceCurve2.png}
    \caption{Typical curve from shearing test.}
    \label{fig:Teori:ShearingTestCurve}
\end{figure}

%%%%%%%%%%%%%%%%%%%%%%%%%%%%%%%%%%%%%

\begin{figure}[H]
\captionsetup[subfigure]{justification=Centering}

\begin{subfigure}[t]{0.35\textwidth}
    \includegraphics[width=5.5cm]{Figurer/Method/Experiment1-6.png}
    \caption{}
\end{subfigure}\hspace{\fill} % maximize horizontal separation
\begin{subfigure}[t]{0.6\textwidth}
    \includegraphics[width=9cm]{Figurer/Method/Method-experiments.png}
    \caption{}
\end{subfigure}
\caption{Flow charts showing a) the overview of the experiments 1-6 and b) the burger preparation.}
\label{fig:Method:FlowChart}
\end{figure}


Results:

\begin{figure}
\begin{subfigure}{.5\textwidth}
  \centering
  \includegraphics[width=0.96\linewidth]{Figurer/Results/TPA/Springiness/SpringinessExp2.png}
  \caption{}
  \label{fig:TPA:Springiness:Exp2}
\end{subfigure}%
\begin{subfigure}{.5\textwidth}
  \centering
  \includegraphics[width=0.96\linewidth]{Figurer/Results/TPA/Cohesiveness/CohesivenessExp2.png}
  \caption{}
  \label{fig:TPA:Cohesiveness:Exp2}
\end{subfigure}
\caption{Springiness (a) and cohesiveness (b) of experiment 2.}
\label{fig:fig}
\end{figure}

\begin{figure}[H]
    \centering
    \begin{floatrow}
    \captionsetup{justification=raggedright,
    singlelinecheck=false}
    
      \ffigbox[\FBwidth]{\caption{...}
      \label{fig:exp1}}
      {\includegraphics[width=8cm]{Figurer/Results/ShearForce/ShearForce-Exp1.png}}

      \ffigbox[\FBwidth]{\caption{...}
      \label{fig:exp2}}
      {\includegraphics[width=8cm]{Figurer/Results/ShearForce/ShearForce-Exp2.png}}
    \end{floatrow}
\end{figure}

\begin{figure}
    \centering
    \includegraphics[scale=0.75]{Figurer/Method/Experiment1-6.png}
    \caption{Experiment 1-6.}
    \label{fig:Method:Experiment1-6}
\end{figure}
\begin{figure}
    \centering
    \includegraphics[scale=0.75]{Figurer/Method/Method-experiments.png}
    \caption{Burger preparation.}
    \label{fig:Method:BurgerPreparation}
\end{figure}

\begin{figure}[H]
    \centering
    \begin{floatrow}
    \captionsetup{justification=raggedright,
    singlelinecheck=false}
    
      \ffigbox[\FBwidth]{\caption{...}
      \label{fig:TPA:Cohesiveness:Exp2}}
      {\includegraphics[width=8cm]{Figurer/Results/TPA/Cohesiveness/CohesivenessExp2.png}}

      \ffigbox[\FBwidth]{\caption{...}
      \label{fig:TPA:Springiness:Exp2}}
      {\includegraphics[width=8cm]{Figurer/Results/TPA/Springiness/SpringinessExp2.png}}
    \end{floatrow}
\end{figure}


%IMPOSSIBLE BURGER:
%Water, Soy Protein Concentrate, Coconut Oil, Sunflower Oil, Natural Flavors, 2% Or Less Of: Potato Protein, Methylcellulose, Yeast Extract, Cultured Dextrose, Food Starch Modified, Soy Leghemoglobin, Salt, Mixed Tocopherols (Antioxidant), Soy Protein Isolate, Vitamins and Minerals (Zinc Gluconate, Thiamine Hydrochloride (Vitamin B1), Niacin, Pyridoxine Hydrochloride (Vitamin B6), Riboflavin (Vitamin B2), Vitamin B12).

\section{METODE}:

Utifra hva Orklas oppskrift, tenker jeg følgende mulige ingredienser:\\
- Soy protein isolate, 500~g, high-moisture extrusion \\
- Soy protein isolate, 500~g, low-moisture extrusion \\
- Soy protein concentrate, 500~g \\
- water \\
- liquid fat: rape and sunflower oil \\
- solid fat: coconut oil \\
- binder: methyl cellulose \\
- binder: seaweed from SINTEF \\

1) Rehydration of TVP, soy isolate or concentrate with boiling/very hot water for 10~minutes. (Fluff with a fork.) \\
2) Add all ingredients in food processor. \\
4) Pulse in food processor until mixed, but not smooth. \\
5) Knead for 2~minutes. \\
6) Form burger patties. \\
7) Wrap in parchment paper, then aluminium foil. (OBS! Wrap tightly to get best texture.  \\
8) Steam burger for 1.5~hours. \\
9) Remove paper, foil and let cool for 10~minutes. \\
10) Grill, cook, marinate. \\

%%%%%%%%%%%%%%%%%%%%%%%%%%%%%%%%%%%%%%%%%%%%%%%%
% From APPENDIX:
\section{Texture analysis}

\subsection{Cook loss}
\begin{table}[H]
    \caption{Results of cook loss.}
    \centering
    %\resizebox{\columnwidth}{!}{
    \begin{tabular}{llll}
\toprule

\textbf{Type} & \textbf{\% cook loss} & \textbf{Type} & \textbf{\% cook loss}\\

\hline

C-0\%       & 9.6 $\pm$ 0.3  & MC-G (0.25\%-0.75\%)     & 6.6 $\pm$ 0.5 \\
MC-0.5\%    & 9.3 $\pm$ 0.3  & MC-G (0.5\%-0.5\%)       & 8.5 $\pm$ 0.8 \\
MC-1\%      & 9.2 $\pm$ 0.5  & MC-G (0.75\%-0.25\%)     & 8.5 $\pm$ 0.5 \\
G-0.5\%     & 7.5 $\pm$ 0.6  & MC-M (0.25\%-0.75\%)     & 5.3 $\pm$ 0.3 \\
G-1\%       & 7.8 $\pm$ 0.3  & MC-M (0.5\%-0.5\%)       & 7.1 $\pm$ 0.13 \\
M-0.5\%     & 10.9 $\pm$ 0.8 & MC-M (0.75\%-0.25\%)     & 9.4 $\pm$ 0.5 \\ 
M-1\%       & 7.6 $\pm$ 0.8  \\
E-1\%       & 9.5 $\pm$ 0.9  & H-1\%       & 6.9 $\pm$ 0.3  \\
E-2\%       & 9.4 $\pm$ 0.3 & H-2\%       & 8.8 $\pm$ 0.3  \\

\bottomrule

    \end{tabular}%}
    \label{tab:ProximateAnalyis:CookLoss}
\end{table}

%%%%%%%%%%%%%%%%%%%%% 
\begin{figure}[H]
    \centering
    \includegraphics[scale=0.4]{Figurer/TVP/Extrusion_zhang_changes_2019.PNG}
    \caption{Extrusion process with low moisture (a) and high moisture (b). Retrieved from \cite{zhang_changes_2019}.}
    \label{fig:Extrusion_zhang_changes_2019}
\end{figure}

\begin{table}[H]
    \caption{Protein, fat and dry matter composition of soy burgers without binder (C-0\%), with 1\% methyl cellulose as binder (MC-1\%), and expected values of a raw burger based on the ingredients used in the food matrix. C-0\% and MC-1\% were analysed both in raw state and fried at 250~W for 4~min. The data is given in \% of the total wet weight of a 40~g burger.}
    \centering
    %\resizebox{\columnwidth}{!}{
    \begin{tabular}{c|ccc}
\toprule

\textbf{Sample} &  \textbf{\% protein} & \textbf{\% fat} & \textbf{\% dry matter} \\

\hline

Raw C-0\%       & 24.3  & 8.9  & 36.3   \\
Raw MC-1\%      & 24.1  & 11.2 & 38.2   \\
Fried C-0\%     & 26.5  & 9.2  & 38.3   \\
Fried MC-1\%    & 24.8  & 9.4  & 39.7   \\
\hline
Expected values for raw burger & 25.2 & 8.1 & 37.5 \\ 

\bottomrule

    \end{tabular}%}
    \label{tab:results:ProximateAnalyis:CompositionOfBurger:C&MC}
\end{table}

%%%%%%%%%%%
% burger formulation:
\begin{figure}[H]
    \centering
    \includegraphics[scale=0.6]{Figurer/Method/Ingredients4.png}
    \caption{Formulation of a 40~g soy burger. The binder was added as a percentage of the total wet weight of a burger. For example: 1\% binder = 0.04~g. The binder amount did not increase 2\%, and was therefore assumed small enough to not change the amount of water, SPI, or oil accordingly.}
    \label{fig:Method:PieDiagram:IngredientOverview}
\end{figure}