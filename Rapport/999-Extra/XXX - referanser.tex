
\
%%%%%%%%%%%%%%%%%%%%%%%%%%%%%%%%%%%%%%%%%%%%%%%%%%%%
                    %EGET ARBEID%
%%%%%%%%%%%%%%%%%%%%%%%%%%%%%%%%%%%%%%%%%%%%%%%%%%%%

@incollection{dahl_prosjektoppgave_2021,
	title = {\textit{{E}xchanging carrageenan with {A}laria esculenta or {S}accharina latissima in pork liver pâté - effect on texture} (unpublished)},
	abstract = {The goal of the project was to analyse if substitution of carrageenan with dried Alaria esculenta (AE) or Saccharina latissima (SL) in pork liver pâté could lead to the same texture. The seaweeds were slightly processed in different ways before adding them in the pâté.},
	author = {Dahl, Sunniva Bauck},
	month = dec,
	year = {2021},
	publisher = {{TBT4500} - {B}iotechnology, {S}pecialization {P}roject. Department of Biotechnology, Food Science, NTNU - Norwegian University of Science {and} Technology.}
}

%%%%%%%%%%%%%%%%%%%%%%%%%%%%%%%%%%%%%%%%%%%%%%%%%%%%
                    %INTRO%
%%%%%%%%%%%%%%%%%%%%%%%%%%%%%%%%%%%%%%%%%%%%%%%%%%%%


%%%%%%%%%%%%%%%%%%%%%%%%%%%%%%%%%%%%%%%%%%%%%%%%%%%%
                    %ALGINATE
%%%%%%%%%%%%%%%%%%%%%%%%%%%%%%%%%%%%%%%%%%%%%%%%%%%%
@incollection{helgerud_alginates_2009,
	title = {Alginates},
	isbn = {978-1-4443-1472-4},
	url = {https://onlinelibrary.wiley.com/doi/abs/10.1002/9781444314724.ch4},
	abstract = {Summary This chapter contains sections titled: Introduction Production Chemical composition Functional properties Gel formation techniques Applications Thickening and stabilising Dairy products Film formation Encapsulation Other applications Summary},
	pages = {50--72},
	booktitle = {Food Stabilisers, Thickeners and Gelling Agents},
	publisher = {John Wiley \& Sons, Ltd},
	editor = {Alan Imeson},
	author = {Helgerud, Trond and Gåserød, Olav and Fjæreide, Therese and Andersen, Peder O. and Larsen, Christian Klein},
	date = {2009},
	doi = {10.1002/9781444314724.ch4},
	keywords = {alginate biosynthesis, alginate use in food applications, alginates, alginic acid - extracted from brown seaweed, block structure of alginates and forming gels, chemical composition - alginates, gel formation techniques, high-molecular-weight polymers, industrially utilised brown seaweeds, long and slow process, natural, sodium alginate hydrates in cold or hot water to giving viscous solutions, thickening/stabilising with {PGA}},
}

%Ingergaard
@article{indergaard_tang_2010,
	title = {Tang og tare - i hovedsak norske brunalger: {Forekomster}, forskning og anvendelse},
	shorttitle = {Tang og tare - i hovedsak norske brunalger},
	url = {https://ntnuopen.ntnu.no/ntnu-xmlui/handle/11250/228180},
	language = {nor},
	urldate = {2022-02-02},
	author = {Indergaard, Mentz},
	year = {2010}
}



%Seasonal variations
@article{schiener_seasonal_2015,
	title = {The seasonal variation in the chemical composition of the kelp species {Laminaria} digitata, {Laminaria} hyperborea, {Saccharina} latissima and {Alaria} esculenta},
	volume = {27},
	issn = {1573-5176},
	url = {https://doi.org/10.1007/s10811-014-0327-1},
	doi = {10.1007/s10811-014-0327-1},
	abstract = {The seasonal chemical profiling of kelp species has historically either being carried out on only a single species or the data dates back over 60 years. This research highlights a detailed chemical composition profile of the four kelp species Laminaria digitata, Laminaria hyperborea, Saccharina latissima and Alaria esculenta over a 14-month period. These kelp species were selected due to their identified potential for cultivation. They were chemically characterised to identify seasonal variations and predict best harvest times. Components of interest included the carbohydrates cellulose, laminarin, alginate and mannitol as well as proteins, ash, metals, moisture, polyphenolics, total carbon and nitrogen content. The highest yields of lamianrin and mannitol coincided with the lowest yields in ash, protein, moisture and polyphenols. The implications of these observations for use of kelp species as a fermentation substrate are discussed.},
	language = {en},
	number = {1},
	urldate = {2022-02-02},
	journal = {Journal of Applied Phycology},
	author = {Schiener, Peter and Black, Kenneth D. and Stanley, Michele S. and Green, David H.},
	month = feb,
	year = {2015},
	pages = {363--373},
	file = {Springer Full Text PDF:C\:\\Users\\sunni\\Zotero\\storage\\S48X6BZN\\Schiener et al. - 2015 - The seasonal variation in the chemical composition.pdf:application/pdf},
}

%%%%%%%%%%%%%%%%%%%%%%%%%%%%%%%%%%%%%%%%%%%%%%%%%%%%
                    %CARRAGEENAN%
%%%%%%%%%%%%%%%%%%%%%%%%%%%%%%%%%%%%%%%%%%%%%%%%%%%%
@incollection{blakemore_carrageenan_2009,
	title = {Carrageenan},
	isbn = {978-1-4443-1472-4},
	url = {https://onlinelibrary.wiley.com/doi/abs/10.1002/9781444314724.ch5},
	abstract = {Summary This chapter contains sections titled: Introduction Raw materials Manufacturing Regulation Structure Functional properties Food applications},
	pages = {73--94},
	booktitle = {Food Stabilisers, Thickeners and Gelling Agents},
	publisher = {John Wiley \& Sons, Ltd},
	editor = {Alan Imeson},
	author = {Blakemore, William R. and Harpell, Alan R.},
	date = {2009},
	doi = {10.1002/9781444314724.ch5},
	keywords = {carrageenan from red seaweeds, carrageenan gelation mechanism, carrageenan in modern-day formulations, carrageenans having a backbone of galactose, functional properties, Lambda carrageenan - least utilised of carrageenans, processed Eucheuma seaweed ({PES}), red seaweeds (Rhodophyceae) as foods in Far East and Europe, Scientific Committee for Food ({SCF}), {US} Food and Drug Administration - no distinction between carrageenan and {PES}},
}

%%%%%%%%%%%%%%%%%%%%%%%%%%%%%%%%%%%%%%%%%%%%%%%%%%%%
                    %METHYL CELLULOSE%
%%%%%%%%%%%%%%%%%%%%%%%%%%%%%%%%%%%%%%%%%%%%%%%%%%%%

@incollection{cash_cellulose_2009,
	title = {Cellulose Derivatives},
	isbn = {978-1-4443-1472-4},
	url = {https://onlinelibrary.wiley.com/doi/abs/10.1002/9781444314724.ch6},
	abstract = {Summary This chapter contains sections titled: Introduction Raw materials and processing Composition and chemistry Food applications Future developments},
	pages = {95--115},
	booktitle = {Food Stabilisers, Thickeners and Gelling Agents},
	publisher = {John Wiley \& Sons, Ltd},
	editor = {Alan Imeson},
	author = {Cash, Mary Jean and Caputo, Sandra J.},
	date = {2009},
	doi = {10.1002/9781444314724.ch6},
	keywords = {anionic cellulose derivative, carboxymethyl cellulose ({CMC}) - water-soluble, cellulose and repeating cellobiose units, cellulose derivatives, cellulose derivatives from cellulose, cellulose gum - in stabilizing low {pH} protein beverages, cellulose gum for corn tortilla manufacture, hydrocolloids in beverages to improve mouthfeel, hydroxypropyl cellulose ({HPC}) - nonionic water-soluble cellulose, methyl cellulose and methylhydroxypropyl cellulose, rheology of cellulose gum solutions, two anhydroglucose units ({AGUs})},
}

%%%%%%%%%%%%%%%%%%%%%%%%%%%%%%%%%%%%%%%%%%%%%%%%%%%%
                    %SOY%
%%%%%%%%%%%%%%%%%%%%%%%%%%%%%%%%%%%%%%%%%%%%%%%%%%%%

@Article{messina_role_2010,
AUTHOR = {Messina, Mark and Messina, Virginia},
TITLE = {The {R}ole of {S}oy in {V}egetarian {D}iets},
JOURNAL = {Nutrients},
VOLUME = {2},
YEAR = {2010},
NUMBER = {8},
PAGES = {855--888},
URL = {https://www.mdpi.com/2072-6643/2/8/855},
PubMedID = {22254060},
ISSN = {2072-6643},
ABSTRACT = {In recent years however, soyfoods and specific soybean constituents, especially isoflavones, have been the subject of an impressive amount of research. There is particular interest in the role that soyfoods have in reducing risk of heart disease, osteoporosis and certain forms of cancer. However, the estrogen-like effects of isoflavones observed in animal studies have also raised concerns about potential harmful effects of soyfood consumption. This review addresses questions related to soy and chronic disease risk, provides recommendations for optimal intakes, and discusses potential contraindications. As reviewed, the evidence indicates that, with the exception of those individuals allergic to soy protein, soyfoods can play a beneficial role in the diets of vegetarians. Concerns about adverse effects are not supported by the clinical or epidemiologic literature. Based on the soy intake associated with health benefits in the epidemiologic studies and the benefits noted in clinical trials, optimal adult soy intake would appear to be between two and four servings per day.},
DOI = {10.3390/nu2080855}
}


@article{nishinari_soy_2014,
	title = {Soy proteins: {A} review on composition, aggregation and emulsification},
	volume = {39},
	issn = {0268-005X},
	shorttitle = {Soy proteins},
	url = {https://www.sciencedirect.com/science/article/pii/S0268005X14000319},
	doi = {10.1016/j.foodhyd.2014.01.013},
	abstract = {Composition of soybean proteins is briefly described. Gels and gelling processes of soybean proteins and other functionalities such as colloidal properties and emulsifying properties are described. The effects of temperature, pH, ionic strength, processing conditions such as high pressure, ultrasonic treatment, utilisation of enzyme, chemical modification are also described since they have been found useful to improve the processing and final product.},
	language = {en},
	urldate = {2022-01-18},
	journal = {Food Hydrocolloids},
	author = {Nishinari, K. and Fang, Y. and Guo, S. and Phillips, G. O.},
	month = aug,
	year = {2014},
	keywords = {Emulsion, Gel, Process, Protein, Soy},
	pages = {301--318}
}

@article{toribio-mateas_impact_2021,
	title = {Impact of {Plant}-{Based} {Meat} {Alternatives} on the {Gut} {Microbiota} of {Consumers}: {A} {Real}-{World} {Study}},
	volume = {10},
	issn = {2304-8158},
	shorttitle = {Impact of {Plant}-{Based} {Meat} {Alternatives} on the {Gut} {Microbiota} of {Consumers}},
	url = {https://www.mdpi.com/2304-8158/10/9/2040},
	doi = {10.3390/foods10092040},
	abstract = {Based on our findings, we concluded that the occasional replacement of animal meat with PBMA products seen in flexitarian dietary patterns can promote positive changes in the gut microbiome of consumers.},
	language = {en},
	number = {9},
	urldate = {2022-01-26},
	journal = {Foods},
	author = {Toribio-Mateas, Miguel A. and Bester, Adri and Klimenko, Natalia},
	month = sep,
	year = {2021},
	publisher = {Multidisciplinary Digital Publishing Institute},
	keywords = {flexitarian, flexitarianism, gut microbiome, gut microbiota, meat alternatives, meat substitutes, plant protein, plant-based diets, plant-based meat alternatives, ultra-processed foods},
	pages = {2040}
}

@article{kyriakopoulou_functionality_2021,
	title = {Functionality of {Ingredients} and {Additives} in {Plant}-{Based} {Meat} {Analogues}},
	volume = {10},
	issn = {2304-8158},
	url = {https://www.ncbi.nlm.nih.gov/pmc/articles/PMC7999387/},
	doi = {10.3390/foods10030600},
	abstract = {In this review, we discuss key ingredients for the production of these novel products, with special focus on protein sources, and underline the importance of ingredient functionality. Our observation is that structuring processes are optimized based on ingredients that were not originally designed for meat analogues applications. Therefore, mixing and blending different plant materials to obtain superior functionality is for now the common practice. We observed though that an alternative approach towards the use of ingredients such as flours, is gaining more interest. The emphasis, in this case, is on functionality towards use in meat analogues, rather than classical functionality such as purity and solubility. Another trend is the exploration of novel protein sources such as seaweed, algae and proteins produced via fermentation (cellular agriculture).},
	number = {3},
	urldate = {2022-01-13},
	journal = {Foods},
	author = {Kyriakopoulou, Konstantina and Keppler, Julia K. and van der Goot, Atze Jan},
	month = mar,
	year = {2021},
	pmid = {33809143},
	pmcid = {PMC7999387},
	pages = {600}
}

% Extrusion process of soy protein
@article{zhang_changes_2019,
	title = {Changes in conformation and quality of vegetable protein during texturization process by extrusion},
	volume = {59},
	issn = {1040-8398},
	url = {https://doi.org/10.1080/10408398.2018.1487383},
	doi = {10.1080/10408398.2018.1487383},
	abstract = {This review aims to summarize the development and current status of food extrusion technology for TVP production and give detailed descriptions about the conformational changes of the main components during the extrusion process, focusing on the effects of barrel temperature, moisture content, feed rate and screw speed on TVP quality. Lastly, we discuss approaches to characterize the extrusion process and propose a new system analysis model.},
	number = {20},
	urldate = {2022-01-26},
	journal = {Critical Reviews in Food Science and Nutrition},
	author = {Zhang, Jinchuang and Liu, Li and Liu, Hongzhi and Yoon, Ashton and Rizvi, Syed S. H. and Wang, Qiang},
	month = nov,
	year = {2019},
	pmid = {29894200},
	publisher = {Taylor \& Francis},
	keywords = {characterization approaches, conformational changes, extrudate quality, extrusion parameters, extrusion process, Texturized vegetable protein},
	pages = {3267--3280},
}

%Challenges PBMA vs meat
@article{sha_plant_2020,
	title = {Plant protein-based alternatives of reconstructed meat: {Science}, technology, and challenges},
	volume = {102},
	issn = {0924-2244},
	shorttitle = {Plant protein-based alternatives of reconstructed meat},
	url = {https://www.sciencedirect.com/science/article/pii/S0924224420304830},
	doi = {10.1016/j.tifs.2020.05.022},
	abstract = {Literature search and supermarket surveys are conducted to identify processing technologies, product formulations, and the chemistry and functionality of various additives applied in meat alternatives production. Comparisons are made between muscle and legume proteins to elucidate disparities in macroscopic aggregation properties that may be greatly diminished through fabrication and ingredient innovation. Due to the highly formulated and processed nature, the nutrition, health, and safety of plant-based meat alternatives are analyzed. 
	Thermoextrusion is found to be the principal reconstructuring technique for meat-like fiber synthesis from plant proteins. Soy and pea proteins, gluten, and polysaccharides are the major building blocks. Through physicochemical interactions, plant proteins are able to aggregate into particles and anisotropic fibrils to impart meat-like texture and mouthfeel. Vegetable oil blends and spices are used to modify the texture and flavor; pigments are added to impart color; vitamins, minerals, antioxidants, and antimicrobials are incorporated to boost nutrition and improve shelf-life. Opportunities exist to overcome technology obstacles and nutrition and safety challenges in further developing the alternatives market.},
	language = {en},
	urldate = {2022-01-27},
	journal = {Trends in Food Science \& Technology},
	author = {Sha, Lei and Xiong, Youling L.},
	month = aug,
	year = {2020},
	keywords = {Meat analogues, Ingredient functionality, Meat alternatives, Meat flavor, Plant-based products, Proteins},
	pages = {51--61}
}

%Expectations
@article{schouteten_emotional_2016,
	title = {Emotional and sensory profiling of insect-, plant- and meat-based burgers under blind, expected and informed conditions},
	volume = {52},
	issn = {0950-3293},
	url = {https://www.sciencedirect.com/science/article/pii/S0950329316300556},
	doi = {10.1016/j.foodqual.2016.03.011},
	abstract = {The use of edible insects as a potential component of food products is gathering interest among scientists, policy makers and the food industry. Although recent research suggests that a growing number of Western consumers might be willing to consume food products containing edible insects or insect-based protein, little is known about the influence of ingredient information on product evaluation. The aim of this study was to examine (i) the overall liking, perceived quality and nutritiousness, and (ii) the emotional and sensory profiling of three commercially available burgers (insect-based, plant-based and meat-based), under blind, expected and informed conditions. In total, 97 young adults took part in this experiment, divided into two sessions to assess the effect of blind tasting. The findings of the study revealed that although the overall liking for the insect burger was comparable to the liking for the plant-based burger, further product development is needed to improve its sensory quality. Complete assimilation occurred for the insect-based burger, which shows that information influenced overall liking. In addition, the informed condition had little influence on emotional conceptualisations. Future research should further explore different informational strategies in order to obtain a better understanding on Western consumers’ evaluation of insect-based products.},
	language = {en},
	urldate = {2022-01-28},
	journal = {Food Quality and Preference},
	author = {Schouteten, Joachim J. and De Steur, Hans and De Pelsmaeker, Sara and Lagast, Sofie and Juvinal, Joel G. and De Bourdeaudhuij, Ilse and Verbeke, Wim and Gellynck, Xavier},
	month = sep,
	year = {2016},
	keywords = {Consumer, Emotion, Expectation, Insect, Liking, Meat, Sensory, Vegetarian},
	pages = {27--31},
	file = {ScienceDirect Full Text PDF:C\:\\Users\\WorkStation 3000\\Zotero\\storage\\Y8STTESM\\Schouteten et al. - 2016 - Emotional and sensory profiling of insect-, plant-.pdf:application/pdf;ScienceDirect Snapshot:C\:\\Users\\WorkStation 3000\\Zotero\\storage\\XHRYFD9F\\S0950329316300556.html:text/html},
}

%%%%%%%%%%%%%%%%%%%%%%%%%%%%%%%%%%%%%%%%%%%%%%%%%%%%
            % RESULTS AND DISCUSSION %
%%%%%%%%%%%%%%%%%%%%%%%%%%%%%%%%%%%%%%%%%%%%%%%%%%%%

% Methyl cellulose experiment 
@article{wi_evaluation_2020,
	title = {Evaluation of the {Physicochemical} and {Structural} {Properties} and the {Sensory} {Characteristics} of {Meat} {Analogues} {Prepared} with {Various} {Non}-{Animal} {Based} {Liquid} {Additives}},
	volume = {9},
	copyright = {http://creativecommons.org/licenses/by/3.0/},
	issn = {2304-8158},
	url = {https://www.mdpi.com/2304-8158/9/4/461},
	doi = {10.3390/foods9040461},
	abstract = {Meat analogue was prepared by blending together textured vegetable protein (TVP), soy protein isolate (SPI), and other liquid additives. Physicochemical (rheological properties, cooking loss (CL), water holding capacity (WHC), texture and color), structural (visible appearance and microstructure), and sensory properties were evaluated. Higher free water content of meat analogue due to water treatment resulted in a decrease in viscoelasticity, the highest CL value, the lowest WHC and hardness value, and a porous structure. Reversely, meat analogue with oil treatment had an increase in viscoelasticity, the lowest CL value, the highest WHC and hardness value, and a dense structure due to hydrophobic interactions. SPI had a positive effect on the gel network formation of TVP matrix, but lecithin had a negative effect resulting in a decrease in viscoelasticity, WHC, hardness value and an increase in CL value and pore size at microstructure. The results of sensory evaluation revealed that juiciness was more affected by water than oil. Oil treatment showed high intensity for texture parameters. On the other hand, emulsion treatment showed high preference scores for texture parameters and overall acceptance.},
	language = {en},
	number = {4},
	urldate = {2022-01-26},
	journal = {Foods},
	author = {Wi, Gihyun and Bae, Junhwan and Kim, Honggyun and Cho, Youngjae and Choi, Mi-Jung},
	month = apr,
	year = {2020},
	publisher = {Multidisciplinary Digital Publishing Institute},
	keywords = {meat analogue, soy protein isolate, emulsion, lecithin, liquid additives},
	pages = {461}
}
