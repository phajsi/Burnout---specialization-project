This thesis investigated the influence of different binders in meat analogue burgers based on soy protein isolate (SPI). 
Meat analogues are becoming increasingly popular as consumers have become aware of the impact of meat production, and the health issues regarding a diet rich in red meat. SPI is used as a base for many meat analogues today, such as Orkla’s Naturli’ burger and the Impossible burger. To mimic meat texture, PBMAs use methyl cellulose (MC).
This polymer has a unique ability to gel and form fibrils upon heating. 
Sodium alginate, extracted from brown algae, can be added to a food matrix with calcium and form a gel. To the best of the author’s knowledge, no research has been published regarding implementing alginate in PBMAs. 
This thesis aimed to see if complete or partial substitution of sodium alginate or seaweed with methyl cellulose could lead to the same texture as burgers with only methyl cellulose.\\

The thesis consisted of six experiments and two sub-experiments. The two sub-experiments were performed to observe the behaviour of the given alginate – one with high G-block fraction (G) and one with low (M) – and the given six different seaweed samples. The alginates gelled as expected in sub-experiment \textit{a}. G yielded a smoother gel than M, most likely due to its higher G-block fractions interacting better with the added calcium and TSPP. 
%The seaweed samples had been analysed in pork liver pâté in the fall of 2021 as a substitution for carrageenan. 
The seaweeds analysed in sub-experiment \textit{b} were from \textit{Saccharina latissima} (SL) and \textit{Alaria esculenta} (AE), which had been through different pre-treatments. The two SL seaweeds – one enzyme pre-treated (E) and one treated with water at high pH (H) – were chosen to investigate in this thesis. This choice was based on the observed higher viscosity.\\

The burgers were mixed and pressed manually using a hamburger press. 
The burgers formed weighed approximately 40~g and had a diameter of 6.5~cm. 
After pressing, the burgers rested for a chosen amount of time, at a chosen temperature, depending on the experiment. 
The burgers were then fried at a chosen temperature until the core temperature was 75°C. 
After resting for 30~minutes, the burgers were cut into pieces and analysed with a shearing test and TPA. The hardness of the sample was obtained from the shearing test and the TPA. 
Cohesiveness, springiness, and chewiness were also obtained from the TPA.
A calibration of the shearing setup was not performed before experiment 6. The shearing test results from the previous experiments had therefore some noise and disturbances, but the results were deemed reliable.

Experiment 1 consisted of developing the food matrix with the right ratios of each ingredient to form a burger that held together. The burger was made of 30.0\% SPI, 3.8\% coconut oil, 3.8\% canola oil, and 62.5\% distilled water. A binder was then added as a percentage of the total weight of the burger. A binder was essential to keep the burger from falling apart. A burger without binder was tested in each experiment (C-0\%), but did not lead to results that were easy to compare with the rest of the burgers. 
C-0\% had to be fried longer than the rest of the burgers for nearly each experiment. The longer frying time resulted in a thicker crust than the other burgers. From experiment 1, the frying time was reduced from 8 to 6~minutes in the pan. \\

Experiments 2 and 3 consisted of optimising the method. Experiment 2 concluded that 1\% of MC and 2\% of MC in the food matrix did not give significantly different textural results from each other, at 5\% level of significance. A higher percentage than 1\% MC was therefore not investigated further in this thesis. Experiment 3 concluded that the burgers should be dry-fried at 250~W, and fried in a raw state instead of frozen, as this was easier to handle in the pan. A raw C-0\% burger was also analysed. However, since the commercial plant-based burgers were fried, only fried burgers were investigated for the remaining of the thesis.\\

Experiment 4 investigated the influence of sodium alginate in the soy burgers compared to MC. Alginate did not work as well as MC, regardless of the fraction of G-blocks. The burgers were also quite wet, which most likely inhibited the Maillard reaction by lowering the temperature. The hardness and springiness were higher for 1\% MC than 1\% alginate, regardless of the G-fraction. G and M were not significantly different. \\

Experiment 5 investigated the influence of a reduction in the amount of MC and alginates (G and M) and the influence of pre-treated SL (E and H) at 1\% and 2\%. A reduction in MC and alginate resulted in a fragile burger that easily tore. A reduction in either polymer was deemed undesired for the food matrix and was therefore not investigated further. H resulted in slightly better textural properties than E. 1\% H had textural parameters similar to 1\% MC. This may suggest that the high pH treatment led to higher availability of alginate than the enzyme pre-treatment. \\

Experiment 6 investigated the influence of a mixture of sodium alginate and MC at different ratios in the soy burgers compared to MC. All burgers with a mixture of alginate and MC resulted in significantly similar textural parameters to 1\% MC. 
This suggested that there were interactions between the two polymers. This coincided well with previous studies reporting a synergistic protein network between sodium alginate and MC. The addition of calcium may also have resulted in a decrease in gelation temperature. 
Furthermore, the burgers with the mixture of alginate and MC were the easiest burgers to fry of all the burgers in the thesis. 
%The alginate burgers were also more fragile than the burgers with mixtures of MC and alginate, regardless of the G-fraction. 
However, it was difficult to draw a reliable conclusion for the optimal ratio between these two
polymers, due to no significant difference between the burgers with different ratios of alginate and MC. Nevertheless, the partial substitution of MC gave promising results and should be investigated further.
