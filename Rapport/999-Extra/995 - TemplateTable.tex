
\begin{table}[H]
\caption{Recipe for the soy burger.}
\label{tab:Method:RecipeBurger}
    \centering
    \resizebox{\columnwidth}{!}{
    \begin{tabular}{ll|l}
\toprule

\textbf{} & \textbf{Step} & \textbf{Procedure} \\

\hline

\textbf{Preparing the SPI:}  & 1 & Weigh SPI and distilled water. \\
                            & 2 & Add both components in a bowl (or a suitable container). \\
                            &   & Spare 25-30~g of the water if alginate is used as binder. \\
                            & 3 & Mix gently by pulsing the food mixer for 1~min or until \\
                            &   & everything is mixed. \footnote{A handheld electric mixer from Philips was used to mix the components} \\
                            & 4 & Let rehydrate for 1~hour in a cold room (T = 4 \celsius). \\
                            & 5 & Weigh coconut and canola oil separately in two suitable beakers. \\
                            & 6 & Melt the coconut oil by warming up the beaker gently.\\ 
                            & 7 &  Add both oils in the SPI bowl. \\
\hline
\textbf{Preparing MC:} & 8a & Weigh MC. \\
\hline
\textbf{Preparing alginate} & 8b & Weigh alginate or seaweed and add it in a 50~mL beaker. \\
\textbf{or seaweed:} & 8b & Stir while adding drops of 96\% EtOH. \\
                    &   & Add drops until alginate or seaweed is dissolved (10-30~drops \\
                    &   & for alginate 20-40~drops for seaweed). \\
                    & 9b & Weigh \ce{CaSO4.H2O} and TSPP. \\
                    & 10b & Slowly add water in the beaker with binder. \\
                    &   & Let stir at maximum speed until binder is dissolved. \\
                    & 11b & Add TSPP in the beaker with alginate. \\
                    & 12b & Dissolve \ce{CaSO4.H2O} in some droplets of distilled water. \\
                    & 13b & Remove stirrer and add \ce{CaSO4.H2O} in the beaker with alginate. \\
                    & 14b & Mix quickly with a spatula. \\
\hline
\textbf{Preparing mix of} & 8c & Follow 7a and 7b-14b for the wanted amounts of respectively \\
\textbf{ MC and alginate:} &   & MC and alginate.\\
\hline
\textbf{Pressing and frying:} & 15 & Gently mix the binder with the rehydrated SPI and fat. \\
                    & 16 & Weigh $\sim$40~g of the raw burger mix in a burger press. \\
                    & 17 & Press burger.\footnote{A non-stick Kitchen Craft mini hamburger press in aluminium was used. Three burgers were pressed for each use. Size: 3 x 6 cm.} \\
                    & 18 & Turn on the oven-top to the right temperature and wait for 10~min.\footnote{The following power levels were studied: 250~W, 500~W and 750~W.} \\
                    & 19 & Fry the burgers at chosen. Flip the burger every 2~min for 4~min or\\
                    &   & until the core temperature reaches 75\celsius. \footnote{Both frying with 2~tsp of canola oil and dry-frying were studied in this thesis.} \\
                    & 20 & Let cool on baking paper for 30~min. \\
\hline
\textbf{Preparing for }  & 21a   & Cut burgers in rectangle pieces 1x2~cm for shear force analysis. \\
\textbf{texture analysis:} & 21b   & Cut burgers in square pieces 1x1~cm for TPA analysis.\\
                            & 22    & Analyse the pieces.\\
\bottomrule
   \end{tabular}
}
\end{table}


%%%%%%%%%%%%%%%%%%%%%%%%%%%%%%%%%%%%%

\begin{center}
\begin{longtable}{l c| l}

\caption{Detailed procedure for burger preparation.} \label{app:fig:RecipeForBurger}\\

\toprule
\textbf{} & \textbf{Step}  & \textbf{Procedure} \\
\hline
\endfirsthead
\multicolumn{3}{c}%
{\tablename\ \thetable\ -- \textit{Continued from previous page}} \\
\toprule
\textbf{} & \textbf{Step}  & \textbf{Procedure} \\
\hline
\endhead
\hline
\multicolumn{3}{r}{\textit{Continued on next page}} \\
\endfoot
\hline
\endlastfoot

\textbf{Preparing the SPI:} & 1 & Weigh SPI and distilled water. \\
                            & 2 & Add both components in a bowl (or a suitable container). \\
                            &   & Spare 25-30~g of the water if alginate is used as binder. \\
                            & 3 & Mix gently by pulsing the food mixer for 1~min or until \\
                            &   & everything is mixed. \footnote{A handheld electric mixer from Philips was used to mix the components} \\
                            & 4 & Let rehydrate for 1~hour in a cold room (T = 4 \celsius). \\
                            & 5 & Weigh coconut and canola oil separately in two suitable beakers. \\
                            & 6 & Melt the coconut oil by warming up the beaker gently.\\ 
                            & 7 &  Add both oils in the SPI bowl. \\
\hline
\textbf{Preparing MC:} & 8a & Weigh MC. \\
\hline
\textbf{Preparing alginate} & 8b & Weigh alginate or seaweed and add it in a 50~mL beaker. \\
\textbf{or seaweed:} & 8b & Stir while adding drops of 96\% EtOH. \\
                    &   & Add drops until alginate or seaweed is dissolved (10-30~drops \\
                    &   & for alginate 20-40~drops for seaweed). \\
                    & 9b & Weigh \ce{CaSO4.H2O} and TSPP. \\
                    & 10b & Slowly add water in the beaker with binder. \\
                    &   & Let stir at maximum speed until binder is dissolved. \\
                    & 11b & Add TSPP in the beaker with alginate. \\
                    & 12b & Dissolve \ce{CaSO4.H2O} in some droplets of distilled water. \\
                    & 13b & Remove stirrer and add \ce{CaSO4.H2O} in the beaker with alginate. \\
                    & 14b & Mix quickly with a spatula. \\
\hline
\textbf{Preparing mix of} & 8c & Follow 8a and 8b-14b for the wanted amounts of respectively \\
\textbf{ MC and alginate:} &   & MC and alginate.\\
\hline
\textbf{Pressing and frying:} & 15 & Gently mix the binder with the rehydrated SPI and fat. \\
                    & 16 & Weigh $\sim$~40~g of the raw burger mix in a burger press. \\
                    & 17 & Press burger.\footnote{A non-stick Kitchen Craft mini hamburger press in aluminium was used. Three burgers were pressed for each use. Size: 3 x 6 cm.} \\
                    & 18 & Turn on the oven-top to the right temperature and wait for 10~min.\footnote{The following power levels were studied: 250~W, 500~W and 750~W.} \\
                    & 19 & Fry the burgers at chosen. Flip the burger every 2~min for 4~min or\\
                    &   & until the core temperature reaches 75\celsius. \footnote{Both frying with 2~tsp of canola oil and dry-frying were studied in this thesis.} \\
                    & 20 & Let cool on baking paper for 30~min. \\
\hline
\textbf{Preparing for }     & 21a   & Cut burgers in rectangular pieces 1x2~cm for shear force analysis. \\
\textbf{texture analysis:}  & 21b   & Cut burgers in square pieces 1x1~cm for TPA analysis.\\
                            & 22    & Analyse the pieces.\\

\bottomrule
\multicolumn{3}{p{15.0cm}}{\textsuperscript{\textit{a}} blablbalbalabalba} \\

\end{longtable}
\end{center}